\documentclass[9pt,a4paper,twocolumn,draft]{article}
\usepackage{graphicx}
\fontsize{9}{20}
\setlength{\columnsep}{20pt}

%opening
\title{\textbf{Generating organic game worlds and mission spaces}}
\author{ \small Nico Andrew Glas \\ \small Amsterdam University of Applied Sciences \\ \small Duivendrectsekade 36-38 \\ \small 1096 AH Amsterdam, The Netherlands \\ \small nico.glas@hva.nl}

\begin{document}
\maketitle  

\begin{abstract}

\end{abstract}

\section{Introduction}
When discussing procedural game content generation research usually focusses on the generation of missions and levels and such. Many games that use a form of content generation use templates (e.g. Blizzard's \emph{Diablo} series) or, ambiguously named, "Dungeon Generation" as done by \emph{Rogue}. Another method for content generation is the use of Voronoi diagrams in level and mission generation. This method is a known technique for the creation of organic game worlds, but has its flaws, especially when the details are observed. In this paper I discuss an improved technique using multiple Voronoi diagrams nested in one and other, making it easier to specify details and creating even more organic game worlds.

\section{Previous and Related Work}

\section{Organic open worlds}

\section{Nesting Voronoi diagrams}

\section{Finding possible paths}

\end{document}
